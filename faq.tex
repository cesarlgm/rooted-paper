\documentclass[a4paper, 11pt]{article}

\input{commandspreamble.tex}

\newcommand{\sigmab}{\boldsymbol\sigma}
\newcommand{\agesigma}{\boldsymbol\sigma_a}
\newcommand{\plim}{\text{plim }}

\definecolor{DarkGreen}{rgb}{0.0, 0.26, 0.15}
\definecolor{LightGreen}{rgb}{0.98, 0.94, 0.9}

\newtheorem{assumption}{Assumption}

\excludecomment{answer}


\onehalfspacing

\begin{document}
	
	\title{JMP: answers to questions I have gotten}	
	\date{{ \normalsize This version: \today  \vspace{.2cm} }}


	\maketitle
	
	\benu
		\item \textbf{Why are the effects so concentrated in a specific range of ages?}
		I do not put a lot of weight on the jump at any given age because these estimates are imprecise. Rather, I want to focus the attention on the fact that there a clear pattern of increase from 6 to 15 years old. 
		
		Two things I need to make clear here:
		\bitem
			\item Are the coefficients from 0 to 6 statistically different from each other?
			\item Can I show that the 6 coefficient is lower than the 15 coefficient.
		\eitem 
		\item \textbf{Why not use exposure to exogenous shocks?}
		
		At some point it was suggested that it was better to use exposure to exogenous shocks as a more credible strategy. For the sake of the argument let's use exposure to climate shocks. 
		
		The idea is the following: women would be hit by climate shocks that induce migration. Timing of migration would be exogenous because migration is driven by the timing of the shock. Then I can compare those migrating early, vs those migrating late:
		
		\beqn
			e_{it}=\delta_{c(i)t}+\beta_1 high_i+\beta_2 late_i+\beta_3high_i\times late_i+\varepsilon_i
		\eeqn
		
		where here $\beta_3$ would give an estimate of the migration the birthplace effect. 
		
		There are two issues here that make this approach difficult with my data:
		\bitem 
			\item How to identify shock-driven migration episodes. If we are thinking about climate, I could restrict the sample to locations and years where the climate shock happened. This would likely substantially reduce my sample. Suppose I limit to 20 percentile shocks. This would likely slash my sample by a fifth, leaving me with about 5000 observations and reducing precision. Still, there is no guarantee that those migration episodes were driven by the shock.
			\item I need a shock that is extensive enough, and with enough frequency so that my sample does not plummet. This only seems feasible with weather shocks.
		\eitem 
		Still, this not solve the endogeneity of the destination choice. 
		
		This might be feasible, but it requires a lot of work. I would only do it if people push back a lot. One thing that I could do is to see the distribution of ages of migration in my data. If I have a generous spread of years, then something with rain might be feasible.
		
		
		\item \textbf{How to think about about the destination?}
		
		One limitation with the baseline specification is that 
		
		\item \textbf{Indonesia experience structural transformation over this period. How does that modify your results?}
		
		I am being very flexible on how the destination enters, and even thought the country has gone through a lot of changes there is a strong persistence in rates of FLFP.
		
		I think there are two ways to look into this:
		\bitem
			\item Show female labor force participation by year and show that even though the level changes, the dispersion persists.
			\item Redo the results by census year. I probably need to bin some age fixed effects.
			\item Results look similar when I change the reference year for the $p_o$
		\eitem 
		
		\item \textbf{What about distance?} 
		
		Here I would like to know two things:
		\bitem
			\item How far these migration episodes happen.
			\item Are there differences in distance of migration across age cohorts.
		\eitem 
		
		\item \textbf{Does industry or occupation matter?}

		\item \textbf{What about the DiD literature?}
	\eenu
	
\end{document}
