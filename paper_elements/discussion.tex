\section{Discussion: why does birthplace matter?}

Here I examine evidence supporting four mechanisms: (i) culture, norms, and learning, (ii) human capital, (iii) marriage and household formation, and (iv) changes in parental investments.


\subsection{Culture, norms, and learning}

Birthplace effects could reflect internalization or learning of local norms and practices around women's work. Epidemiological research sees country of ancestry FLFP rates as a summary measure that captures variations in preferences, beliefs, and culture that influence aggregate female employment and can be passed down through generations \citep{Fernandez2009}. 


In table \ref{table:table_8}, I use data from the Ethnographic Atlas \citep{Murdock1967} to document that FLFP rates capture meaningful variation in cultural practices and gender norms \textit{within Indonesia}. Columns (1) to (3) show results from regressing the regency's FLFP rates on the prevalence of several traditional or pre-modern norms/practices and other regency characteristics. I include as regressors the prevalence of practices related to gender or marriage, namely matrilocality, emphasis on female chastity, bride price, use of plow agriculture, polygamy, and male-only agriculture. Column (1) shows that these variables are highly significant, and they alone account for 30\% in the variation of FLFP rates. Moreover, columns (2) and  (3) show these variables remain jointly significant when including additional controls, such as the regency's industrial and age structures and overall education levels. In addition,  appendix table \ref{table:table_A5} shows these practices predict other female outcomes such as age at first marriage and number of children. In contrast, columns (4) to (6) in  \ref{table:table_8} show that they have little bite when using the regency's male LFP rate as the outcome.

{\onehalfspacing
	\tftext{\input{\thesispath/results/tables/table_8.tex}}
}


Table \ref{table:table_8}  suggests that women born in high-FLFP regencies are exposed to a distinct set of norms and cultural practices that could impact their choices and preferences. Combined with the importance of late childhood and early adolescence in my main results, this aligns well with evidence from psychology and economics that identifies this period as key for preference formation. Early adolescents are mature enough to form their own opinions but receptive to external influences \citep{Markus1986}. For instance, \cite{Dhar2022} find long-lasting effects from interventions targeting gender views of Indian teenagers, and \cite{Olivettib} show that exposure to classmates' working mothers during secondary has long-term effects on women's work decisions in the US.




\subsection{Human capital}
Exposure to birthplace could affect women's labor supply via their career expectations and their educational investment. Exposure to an environment where women actively participate in economic activities could alter their career expectations and make them more likely to invest in education. For example, \cite{Molina2022} show that in high-FLFP Japanese municipalities, young women exhibit greater educational aspirations, leading to increased investment in schooling. 


However, investment in education is unlikely to account for my results. High-FLFP regencies have worse primary and secondary completion rates (see appendix table \ref{table:table_A6}). Moreover, there is little evidence that women who stay longer in these regencies invest more in education. If schooling drove the patterns observed in figure \ref{fig:figure_3}, I should observe increasing persistence when I use schooling measures as the outcome. Appendix figure \ref{fig:figure_B7} shows no evidence of this when the outcome is the likelihood of completing secondary school. Although the figure suggests an apparent increase in the likelihood of completing primary school, these slopes are imprecise, and I cannot reject that all of them are equal (i.e., null birthplace effects).\footnote{I also cannot reject that all slopes from 8 to 17 are the same. Additionally, the \textit{employment} persistence coefficients remain unchanged when I control for interactions between birthplace FLFP, migration age, and completed primary dummies. If higher employment were mainly due to higher completion rates of primary school, the coefficients in figure \ref{fig:figure_3} should flatten once I control for this triple interaction.}



\subsection{Marriage and household formation}

Previous research emphasizes the interaction between husbands' background and women's labor market choices \citep{Fernandez2004,Blau2011}. In appendix figure \ref{fig:figure_B8} I restrict the sample to women with identified husbands in the Intercensal Survey and test whether high-exposure women choose husbands of certain backgrounds. I focus on five main traits: being an internal migrant, born in above-mean-FLFP regency, high-school graduate, employed, and salaried. If high-exposure women were selecting husbands with different backgrounds, there should be clear trends in the slope estimates. The lack of such pattern in both panels of figure \ref{fig:figure_B8} suggests that women with low and high exposure select partners with similar traits.\footnote{I also studied whether women's choices are affected by the length of their husband's exposure to high-FLFP regencies. Nevertheless, the sample is small and I lack the power to draw any meaningful conclusion.}





\subsection{Changes in parental investment}
{\cite{Molina2022} suggests that exposure to local labor market opportunities influences parental investment in girls' education. There are two main ways through which parental investment could explain my results. Although I cannot fully discard these explanations, they do not seem very plausible in my context.} 


{The first explanation is pure selection. The increasing persistence could reflect that parents who stayed longer in high-FLFP regencies happened to invest more in their children. If parents who stayed longer in high-FLFP regencies invested more in their daughters' education, one would expect that girls from these locations came from families with higher socioeconomic backgrounds. However, panel (d) of figure \ref{fig:figure_6} shows little evidence of selection on parental socioeconomic background. Moreover, since high-FLFP regencies have worse educational outcomes, high-investment parents would likely leave these locations earlier rather than later.}

Another possibility is that staying longer in these locations affected parental investment. However, there is little evidence that staying longer in these locations is associated with higher education. Admittedly, investment could act through channels other than schooling, but changes in investment would need to occur at a very specific time in the children's development to account for my results fully.





\section{Robustness}
\label{sec:robustness}


{My results are robust to multiple variations in the estimation strategy. My main estimates limit the sample to women migrating at 17 or younger and source birthplace FLFP from the 2010 Indonesian Census. Section \ref{sec:robustness1} shows that I obtain similar results when I restrict the sample to women migrating up to 16 or up to 18 years old. Section \ref{sec:robustness2} shows I get similar estimates when sourcing the FLFP from the census prior to the Intercensal Survey year.  Section \ref{sec:robustness3} addresses the possibility that early entry to the labor market drives my results.  \purple{Finally, section \ref{sec:marriage} shows evidence against marriage-related migration.}}


\subsection{Maximum age at migration in the sample}
\label{sec:robustness1}
My main results include all women who migrated at 17 or younger. A concern with this sample is that women migrating at 17 or 18 are likelier to consider their job prospects when migrating.

{Appendix table \ref{tab:table_A7} shows results for different maximum migration ages. The table estimates the employment effect of longer stays for two women: one born in a regency at the 75th FLFP percentile and another born at the 25th percentile,  assuming both stayed in their birthplace until age 16. That is, these estimates are the difference between the gaps at 16 and 0-3 years old.}



Changing the maximum migration age has minimal effects on my estimates. Narrowing the sample to 16 or younger in column (1) or widening it to 18 or younger in column (3) generates results close to my baseline (column (2)). Furthermore, the persistence coefficients ($b_a$) from the three samples exhibit similar behavior and are quite similar in magnitude, with the bulk of the increase occurring between 6 and 14 years old. I interpret this as evidence that my results are not driven by different selection patterns for the oldest migrants.

\subsection{Reference year for the birthplace FLFP}
\label{sec:robustness2}

My main results source birthplace FLFP rates from the 2010 Indonesian Census. Although FLFP rates are very persistent (see section \ref{sec:persistence}), the rates in the 2010 census could be a poor proxy for the rates ``experienced'' by the women from the 1985 and 1995 Intercensal Surveys.


Appendix figure \ref{fig:figure_B9} shows that my results are robust changes in the FLFP reference year.
The dark red (filled) circles show estimates when I source birthplace FLFP rates from the census prior to the Intercensal Survey year,\footnote{That is: 1980 census for the 1985 survey, 1990 for 1995, and 2000 for the 2005 Intercensal Survey.} while the orange (hollow) circles show my baseline estimates. The results for both women in panel (a) and men in panel (b) are fairly similar under both strategies.


\subsection{Child labor}
\label{sec:robustness3}

Child labor is a potential concern for my estimates.  While contemporary child labor rates in Indonesia are generally low, this was not true in the 1980s. The share of children aged 10-14 working declined from 11\% in 1980 to approximately 3\% in 2010.\footnote{Information about work is available only for people aged 10 or more.}	Moreover, the strong positive correlation between FLFP and female child labor (FCL) rates raises the possibility that birthplace effects could be driven by early labor market entry (See appendix figure \ref{fig:figure_B10}) 

However, my estimates are robust to controlling for FCL rates. Appendix figure \ref{fig:figure_B11} shows birthplace persistence estimates when controlling for birthplace FCL rates. The baseline estimates in orange (hollow circles) control for regency-year-age fixed effects, a quadratic polynomial on age, and education fixed effects. The estimates in red (filled circles) control for the birthplace FCL rate, while the purple estimates (plus sign markers) add interactions between migration age and the birthplace FCL rate. The estimates are largely unaffected by the inclusion of the child labor rates.


\subsection{Marriage-related migration}
\label{sec:marriage}

Marriage could drive the birthplace persistence in employment if there is an interaction between birthplace, migration age and marriage. Early marriage is associated with worse health and economic outcomes for women \citep{Corno2023}. If women from low-FLFP regencies are more likely to marry and migrate around 12-15 years old, this could explain why they are less likely to work later in life.

Appendix figure \ref{fig:figure_B12} uses detailed IFLS marriage history data to test whether marriage-related migration drives the employment patterns. First, in panel (a), I show the relationship between migration and women's marriage.\footnote{The IFLS collected marriage-history information for women only.} I classify migration episodes as marriage-related if the respondent married the year before, the year, or the year after she migrated. I then regress the marriage-related dummy on migration age fixed effects and interactions between migration age and birthplace FLFP. The plotted interaction estimates in panel (a) show a clear decline in the coefficients, suggesting that women from high-FLFP regencies become less likely to migrate due to marriage the longer they stay in their origin. This could explain the employment patterns I document. 

Nevertheless, panel (b) shows no evidence that selection on marriage-related migration accounts for the employment effects. Panel (b) displays the baseline IFLS estimates (red/hollow circles) along with results that control for interactions between migration age and the marriage-migration dummy and interactions between migration age, the marriage-migration dummy, and birthplace FLFP (orange/filled circles). If selection on marriage migration drove the birthplace effects, the trend should flatten once I account for the marriage motive. Nevertheless, the patterns remain virtually unchanged.











