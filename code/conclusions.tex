
\section{Conclusions}
\label{sec:conclusions}

In this paper, I provide new evidence on the large and persistent geographic variation in women's labor supply within multiple countries at different levels of development. I then focus on Indonesia, a large developing country home to more than 118 million women.

I link childhood exposure to  Indonesia's spatial FLFP inequality and women's adult labor market outcomes. Using the traditional ``epidemiological'' approach from previous literature, I first document that birthplace is highly predictive of the labor supply choices of internal female migrants. Women currently exposed to the same labor market make very different choices when they come from places with different FLFP rates.


I use rich data on migration history to argue that more prolonged exposure to these locations affects women's work choices. By using migration age data, I show that women exposed longer to high-FLFP labor markets are likelier to work as adults than those exposed longer to low-FLFP locations. These effects are large and are driven by exposure during the formative years between the ages of 6 and 14. In all, staying in a location at the 75th FLFP percentile between 6 and 14 makes women five percentage points more likely to work than those staying in a 25th percentile location. The validity of these estimates hinges on the assumption that omitted variable bias is constant across migration age, which is supported by the data.


These results are consistent with the internalization of local gender norms. Longer exposure to high-FLFP locations is also associated with small marriage delays and lower fertility. Moreover, the effects are concentrated during formative ages when norms are malleable.  The data do not support the idea that investment in education or selection based on family background are the main drivers of these results. Nevertheless, additional research is necessary to further understand how local labor markets affect women's choices.




\clearpage
