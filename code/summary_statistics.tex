\subsection{Summary statistics}
\label{Ch1-summary_stats}

Table \ref{table:table_1} gives statistics for the whole sample and by gender and migration status. Columns (1) to (3) highlight three features of the Indonesian labor market. First, internal migration is common, with approximately one-fifth of Indonesians residing outside their birthplace. These internal migrants are my analysis's primary focus, representing a large cross-section of the Indonesian population. Second, the labor market is highly informal and agricultural, with 49\% of workers being self-employed and the same share working in agriculture. Third, there are large gender gaps in types and rates of employment. Women are 38 p.p. less likely to work than men.\footnote{While this is a large gap, it is typical of Southeast Asia} Furthermore, women are five times more likely to be unpaid or family workers. Most unpaid workers work in agriculture (75\%) and the retail industry (19\%). Women are also more likely to work in services and manufacturing.




Columns (4) to (6) zoom in on women, distinguishing between non-migrants, all migrants, and women who migrated at 17 or younger (hereafter young migrants). Compared to non-migrants, female migrants are more educated but less likely to be employed. They are also more likely to be salaried and live in urban areas. Moreover, despite some differences in education, marriage rates, and fertility, young migrants are similar to the typical migrant.

The final rows of table \ref{table:table_1}  provide details on women's migration. Moves are primarily motivated by reasons other than work, with over 85\% associated with education or ``other reasons''. The survey does not break down the ``other'' category, but IFLS data suggests that most of these moves are family-related. The last row summarizes migration distances. On average, migrants undertake long-distance moves covering 687 kilometers (426 miles). Young migrants travel shorter distances, but their moves still span 438 kilometers (272 miles).


Table \ref{table:table_3} characterizes migration flows by urbanicity of the origin and destination regencies. Following \cite{Bryan2019},  I classify regencies according to the share of the regency's population living in areas classified as urban by the Indonesian Central Bureau of Statistics. I label as urban all regencies with an urban share above 43.3\%. 
I chose this cutoff to match the share of people living in urban regencies with the aggregate urban share in the census. Table \ref{table:table_3} shows migration is not exclusive to rural regencies as migration rates are similar in urban and rural regencies. Moreover, migration is not solely rural-to-urban. Panel A breaks down flows by origin and destination urbanicity and shows large urban-to-rural, rural-to-rural, and urban-to-urban flows.\footnote{I observe similarly large flows when splitting regencies at the median FLFP rate. There is movement in all possible FLFP directions: low-low, low-high, high-low, and high-high. It is not the case that women primarily migrate towards high-FLFP locations.} Finally, panel B shows considerable heterogeneity in employment rates within each regency classification. Thus, differences between rural and urban regencies are not the main driver of the dispersion in female labor supply.


\tftext{\input{\thesispath/results/tables/table_1.tex}}


\tftext{\input{../analysis/output/table_main_summ_stats}}


\tftext{\input{\thesispath/results/tables/table_3.tex}}


\tftext{\input{../analysis/output/table_migflows_stats}}
