\section{Introduction}

There are surprisingly large and persistent differences in female labor force participation (FLFP) rates within multiple countries at different levels of development. Figure \ref{fig:figure_1} illustrates the high dispersion in subnational FLFP rates within several developing countries and the United States. In most countries shown, gaps of over 15 percentage points between localities are common.\footnote{Using the interquartile range as a benchmark, the gap between the localities at the 75th and the 25th FLFP percentiles is over 15 p.p. for seven out of the ten countries in the figure. It is 28 p.p. for China, 22 p.p. for Indonesia, and 10 p.p. in the United States.} This large within-country dispersion has generally gone unnoticed in the literature \citep{Charles2018}, and, as a consequence, we know very little about its causes and implications for women's outcomes.  In particular, there is scarce evidence on whether exposure to localities with high or low female labor market participation affects women's work choices in adulthood. %Consequently, we have little understanding of whether these disparities are persistent features of these localities or are transmitted across generations.

%\smartinputtable{/results/figures/figure_1.tex}


In this paper, I use rich data from Indonesian internal female migrants to show that subnational FLFP dispersion strongly affects the labor market outcomes of women born across different Indonesian localities. Specifically, I exploit detailed geographic and migration data to show, conditional on the current place of residence,  women exposed longer to high-FLFP localities are more likely to work when adults.

In this paper, I use rich data from Indonesian internal female migrants to show that subnational FLFP dispersion strongly affects the labor market outcomes of women born across different Indonesian localities.\footnote{Migration is relatively common in Indonesia, with approximately one in five Indonesians residing outside their birth locality.} I identify the birthplace causal effect by leveraging variation from women who live in the same labor market as adults but who left their birthplace at different ages. This approach essentially compares the labor supply of women who migrated in early childhood versus those who left in their early teens. If women born in high-FLFP places are more likely to work the longer they stay there, I attribute this to the effect of longer exposure. Under the assumption that omitted variable bias is constant for women migrating at different ages, this strategy allows me to distinguish the causal effect of the origin labor market from differences in women's characteristics.  In addition, by focusing on the birthplace rather than the destination labor market, I uncover variation more likely to be driven by women's labor supply choices rather than structural labor demand differences across locations.


Indonesia is an ideal setting to study place effects on women's labor supply because it is a large nation with within-country FLFP variation similar to other developing countries. Additionally, Indonesia offers rich representative datasets tracking people's birthplace and current location at a detailed geographic level. My main analyses source data from the 1985, 1995, and 2005 Intercensal Surveys and all waves of the Indonesian Family Life Survey (IFLS). These datasets track respondents' birthplace, current location, and migration history at a geographic level not available in traditional sources from other countries \citep{Bryan2019}. Throughout the paper, I identify localities as Indonesian districts, which are administrative geographies akin to counties in the United States. The average regency in my dataset is approximately twice the size of the US state of Rhode Island and houses eight hundred thousand people.

I find that spending late childhood and early teens in high-FLFP areas makes women more likely to work as adults. Moreover, the longer they stay in these areas, the more likely they are to join the labor force later in life. In my preferred specification, living in a place at the 75th FLFP percentile between the ages of 6 and 14 makes women five percentage points (p.p.) more likely to work than those living in a 25th percentile place. These magnitudes are quantitatively important as they imply that approximately 23\% of the current spatial inequality in FLFP is transmitted to the next generation through birthplace effects. In contrast, I do not find such effects for men. Depending on the specification, staying longer in high-FLFP locations has either no effect or a negative effect on men's employment in adulthood.

These findings highlight the importance of early exposure to the local environment in shaping gender disparities in the labor market, even within the same country. They mirror the persistence in fertility decisions found by \cite{Fernandeza} and \cite{Fernandez2009} for second-generation immigrants to the United States, and they highlight that women's labor supply decisions can be influenced by factors present long before they reach working age.


My results are consistent with a setting where birthplace effects act through internalizing norms around women's work. Using data from the Ethnographic Atlas \citep{Murdock1967}, I show that a regency's FLFP captures variation in pre-modern gender norms within Indonesia. Moreover, the birthplace effects are concentrated during the formative period of late childhood and early adolescence, a time when children's views are still malleable but beginning to solidify \citep{Markus1986}. This aligns well with evidence that children's views about women's standing in society are susceptible to change during these ages \citep{Dhar2022,Olivettib}.

I do not find support for alternative mechanisms highlighted by previous literature: (i) higher investment in schooling, (ii) marriage and household formation, and (iii) changes in parental investment \citep{Molina2022,Fernandez2004,Blau2011}.  There is little evidence indicating that women who are more exposed to high-FLFP locations stay longer in school or that they choose husbands with different characteristics. Moreover, high-FLFP locations have worse schooling completion rates across the board, suggesting that they have lower-quality schooling. In addition, if changes in parental investments were the primary driver behind these outcomes, they would have to occur at a very specific time in the child's life and change the timing of their move to account for my results. While this is possible, it does not seem very likely.

My estimates assume that omitted variable bias is constant across migration age; that is, the correlation between birthplace FLFP and unobserved determinants of women's labor supply is the same regardless of when they left their birthplace. Differences in factors I do not control for between women born in different locations are not sufficient to violate this assumption. For example, women from high-FLFP locations may be more likely to work because their parents had higher resources to invest in their education compared to those from low-FLFP locations. This would create differences between women from different origins that are not driven by birthplace effects. However, such differences do not necessarily violate the constant bias assumption. A violation would require the resource gap to be larger (or smaller) for women migrating at older ages. In the paper, I provide evidence that the gap in resources and other covariates remains fairly constant across migration ages, thereby supporting my identification assumption.


This paper first lies at the intersection of the culture and the place effects literature. Research using the ``epidemiological approach'' to culture shows that FLFP and cultural proxies from the countries of ancestry can predict immigrants' work and fertility decisions \citep{Fernandez2004,Fernandez2009,Fernandez2013,Nollenberger2016a}. I contribute to this literature by adapting the ``epidemiological approach'' to study the determinants of within-country variation in women's work choices in a large developing country. Additionally, I extend it using techniques inspired by the place effects literature \citep{Chetty2018, Chetty2018a, Milsom2021} to show that exposure to one's place of birth during a critical developmental period can continue influencing women's choices even after the exposure ceased. This complements existing evidence showing that \textit{current} exposure to labor markets can affect women's expectations, labor supply, and educational investment \citep{Molina2022, Boelmann, Milsom2021,Moreno-Maldonado2019}, while also highlighting the rich within-country variation in the determinants of women's choices recently noted in the literature \citep{Charles2018,Boelmann}.


More broadly, this paper also contributes to the literature showing that where people grow up and live has important implications for intergenerational mobility \citep{Chetty2018,Chetty2018a}, racial inequality \citep{Chetty2020}, human capital accumulation \citep{Molina2022}, criminal activity \citep{Damm2014}, and political behavior \citep{Brown2023}. I add to this literature by providing new evidence linking women's birthplace to their adult outcomes in a large developing country. This complements existing work showing that spatial inequality is particularly important for women's human capital investment in West Africa \citep{Milsom2021}.

Finally, my paper contributes to the vast literature on the determinants of women's labor supply. This research has primarily exploited cross-country differences in female labor supply to study its determinants and implications \citep{Olivetti2008, Olivetti2014, Blau2020, Blau2015}. In this paper, I document the existence of large and persistent differences in female labor supply within multiple developing countries and explore some of its implications. In this way, my approach aligns more closely with recent literature documenting that factors such as commuting and sexism can explain geographic differences in women's labor supply within the United States and France \citep{Black2014a, Moreno-Maldonado2019, LeBarbanchon2021,Charles2018}.
