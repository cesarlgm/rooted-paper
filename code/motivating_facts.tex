\section{Two Facts about Women's Labor Supply}
\label{sec:motivating_facts}


\subsection{Within-Country Dispersion in Women's Labor Supply is Pervasive Across Countries}


Table \ref{tab:cross-country_main} summarizes within-country variation in women's and men's
employment rates across a set of countries with disaggregated regional data in IPUMS
International.\footnote{Data for the full set of countries are reported in Table
\ref{tab:cross-country_app}. All results discussed here generalize to that broader sample.
Further details on the cross-country data are provided in Appendix \ref{ap:data}.}
I restrict the sample to individuals aged 19--64 and compute employment rates at the lowest
available geographic level, typically administrative units comparable to counties or municipalities.
Countries are ordered by the interquartile range (IQR) of female employment.

Three patterns emerge. First, columns (1) to (3) show that, despite substantial differences in mean
employment rates, all countries exhibit large within-country variation in women's
employment.%\footnote{Appendix Table \ref{tab:cross-country_appendix} shows that this dispersion is not driven by variation in unpaid employment. In Indonesia, an IQR of 12 p.p. remains when focusing on paid employment only, more than twice that of men.} 
In most countries, the gap between the 75th and 25th percentile localities exceeds 15 p.p.

Second, this dispersion is pervasive across levels of development and geographic regions. Table \ref{tab:cross-country_main} includes countries from Asia, the Americas, Africa, and Europe, spanning middle-income economies such as Indonesia and Mexico and high-income economies such as the United States and Spain. This suggests that the underlying forces are not confined to particular regions or income levels.

Third, columns (4) to (6) show that dispersion is concentrated among women. With the exception of Brazil, the United States, and Spain, women's dispersion substantially exceeds men's, and in ten of the eighteen countries it more than doubles it. While men work at relatively high rates across regions, women's employment varies sharply by locality.\footnote{Although local employment rates are measured with error, it is unlikely that measurement error is the primary driver of these patterns. The dispersion in women's employment is consistently much larger than that of men. Explaining the gender gap through measurement error would require implausibly large differences in measurement precision across genders.}

\toWrite{Make a point that this has not really been pointed out in the literature.}


\subsection{The Dispersion of Female Labor Supply is Highly Persistent}
\label{sec:persistence}

Figure \ref{fig:figure_2} zooms in on Indonesia. The map displays district female employment rates across, grouped into quintiles. Darker shades indicate higher employment. The dispersion is widespread and not driven by any particular province, island, or subset of districts.

This dispersion could reflect (i) temporary shocks that depress women's employment in specific regions, (ii) measurement error, or (iii) persistent structural differences across districts. If temporary shocks or measurement error were the main drivers, persistence would be low: shocks dissipate over time, and measurement error should be uncorrelated across decades. By contrast, high persistence points to structural determinants.

Columns (1) to (3) of Table \ref{tab:idn_flfp_persistence} report autocorrelations of district-level employment rates across different horizons. The estimates are high: around 80\% at a ten-year horizon and about 70\% even after thirty years. These magnitudes indicate that structural differences across districts drive the observed dispersion. For comparison, column (4) reports a contemporaneous correlation with men's employment rates of 51\%, implying that women's employment is more strongly correlated with its own past than with men's employment in the same year.\footnote{High persistence is not unique to Indonesia. Appendix Figure \ref{fig:figure_B1} shows that 10-year autocorrelations exceed 67\% in most countries.}

The next sections examine whether exposure to this persistent regional inequality during childhood shapes women's labor supply decisions in adulthood.


\smartinputtable{input/tab/table_cross_country_dispersion_main}

\smartinputtable{input/fig/figure_idn_flfp_map_paper}

\smartinputtable{input/tab/table_idn_flfp_persistence}

%\tftext{\input{\thesispath/results/tables/table_5.tex}}

