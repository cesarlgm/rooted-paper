\section{Two facts about women's labor supply}
\label{sec:motivating_facts}

In this section, I use data from IPUMS International to present two facts on female labor supply. First, I show that large within-country differences in women's labor supply are pervasive worldwide. Next, I zoom in on Indonesia and characterize the large and highly persistent dispersion in female labor supply across regencies.

\subsection{Fact 1: Within-country dispersion in women's labor supply is pervasive across countries}



Table \ref{tab:table_4} provides a snapshot of the within-country variation in women's and men's employment rates for several countries. These countries come from a larger set with disaggregated regional employment data in IPUMS International.\footnote{Data for the complete set of countries is available in table \ref{tab:table_A1}.  All insights discussed in this section generalize to this larger set of countries. Further details about the cross-country data are available in appendix \ref{sec:country_data}} For all countries, I restrict the sample to people aged 18 to 64 and compute the employment rates at the smallest geographical unit available, often corresponding to an administrative unit similar to a county or a municipality. The table orders countries from highest to lowest dispersion in female employment rates, as measured by the interquartile range (IQR) in employment.



This table highlights three insights on women's employment. First, columns (1) to (3) show that, despite the significant differences in the mean, all countries exhibit large variations in women's employment rates \textit{within} their borders.\footnote{Appendix table \ref{tab:table_A2} shows that the within-country dispersion in women's employment is not the result of regional variation in the rates of unpaid employment. For Indonesia,  55\% (IQR 12 p.p.) dispersion remains when I focus on paid employment only. This --reduced--  IQR of 12 p.p. is more than twice that of men's.} For most countries, the gap between the 75th and 25th percentile localities is above 15 percentage points (p.p.).

Second, the large dispersion is widespread across countries at different levels of development and geographic regions. Table  \ref{tab:table_4} includes countries from Asia, the Americas, Africa, and Europe. It also includes middle-income countries like Indonesia and Mexico and high-income countries such as the USA and Spain. These findings suggest that the factors driving this dispersion are not limited to specific regions or income levels.

Third, columns (4) to (6) show that the large within-country dispersion in employment is primarily concentrated among women. With the exceptions of Brazil, the United States, and Spain, women's dispersion is substantially larger than men's. In ten of the seventeen countries, women's dispersion \textit{more than doubles} men's. Therefore, while men work at high rates across all regions within these countries, women's rates vary significantly by locality.\footnote{While the district employment rates are measured with error, I find it unlikely that this is the primary dispersion driver. The variation in women's employment is much larger than men's across most countries. Even if measurement error were greater for women, this difference would have to be substantial to account for the gender differences in table \ref{tab:table_4}.}




\subsection{Fact 2: The dispersion of female labor supply  is highly persistent}
\label{sec:persistence}

Figure \ref{fig:figure_2} zooms in on Indonesia. The map shows women's employment rates in all 268 regencies, grouped by color into quintiles. Darker blues indicate higher employment rates.  
The map highlights that the dispersion in women's employment extends across the whole country and is not driven by any particular province, island, or group of regencies.

This large dispersion in women's employment rates could stem from (i) temporary economic shocks that depress women's employment in some parts of Indonesia, (ii) measurement error,  or (iii) structural differences across regencies that are correlated with employment. We should expect low employment rate persistence if the dispersion is mainly due to temporary shocks or measurement error. This is because temporary shocks should fade over time, and I expect measurement error to be independent across decades. In contrast, high persistence suggests that structural factors are driving the variation.


Columns (1) to (3) of table \ref{table:table_5} show autocorrelation estimates for the regency-level employment rates across different time horizons. The high autocorrelation estimates suggest that structural differences across regencies drive the variation in women's employment rates. They start at 80\% for the ten-year horizon and stay as high as 70\% for the thirty-year horizon. As a benchmark, column (4) reports a simultaneous correlation with men's employment rates of 51\%. This means women's employment rates correlate more with themselves 30 years apart than with men's in the same year.\footnote{The large persistence of female employment rates is not exclusive to Indonesia. Appendix figure \ref{fig:figure_B1} shows that large 10-year auto-correlations also arise in other countries. For most countries,  this auto-correlation is over 67\%.} The following sections analyze whether exposure to this persistent regional inequality during childhood permanently affects women's labor supply choices in adulthood.




\tftext{\input{\thesispath/results/tables/table_4.tex}}

\tftext{\input{\thesispath/results/figures/figure_2.tex}}

\tftext{\input{\thesispath/results/tables/table_5.tex}}

