

\section{Robustness}
\label{sec:robustness}


{My results are robust to multiple variations in the estimation strategy. My main estimates limit the sample to women migrating at 17 or younger and source birthplace FLFP from the 2010 Indonesian Census. Section \ref{sec:robustness1} shows that I obtain similar results when I restrict the sample to women migrating up to 16 or up to 18 years old. Section \ref{sec:robustness2} shows I get similar estimates when sourcing the FLFP from the census prior to the Intercensal Survey year.  Section \ref{sec:robustness3} addresses the possibility that early entry to the labor market drives my results.  \purple{Finally, section \ref{sec:marriage} shows evidence against marriage-related migration.}}


\subsection{Maximum age at migration in the sample}
\label{sec:robustness1}
My main results include all women who migrated at 17 or younger. A concern with this sample is that women migrating at 17 or 18 are likelier to consider their job prospects when migrating.

{Appendix table \ref{tab:table_A7} shows results for different maximum migration ages. The table estimates the employment effect of longer stays for two women: one born in a regency at the 75th FLFP percentile and another born at the 25th percentile,  assuming both stayed in their birthplace until age 16. That is, these estimates are the difference between the gaps at 16 and 0-3 years old.}



Changing the maximum migration age has minimal effects on my estimates. Narrowing the sample to 16 or younger in column (1) or widening it to 18 or younger in column (3) generates results close to my baseline (column (2)). Furthermore, the persistence coefficients ($b_a$) from the three samples exhibit similar behavior and are quite similar in magnitude, with the bulk of the increase occurring between 6 and 14 years old. I interpret this as evidence that my results are not driven by different selection patterns for the oldest migrants.

\subsection{Reference year for the birthplace FLFP}
\label{sec:robustness2}

My main results source birthplace FLFP rates from the 2010 Indonesian Census. Although FLFP rates are very persistent (see section \ref{sec:persistence}), the rates in the 2010 census could be a poor proxy for the rates ``experienced'' by the women from the 1985 and 1995 Intercensal Surveys.


Appendix figure \ref{fig:figure_B9} shows that my results are robust changes in the FLFP reference year.
The dark red (filled) circles show estimates when I source birthplace FLFP rates from the census prior to the Intercensal Survey year,\footnote{That is: 1980 census for the 1985 survey, 1990 for 1995, and 2000 for the 2005 Intercensal Survey.} while the orange (hollow) circles show my baseline estimates. The results for both women in panel (a) and men in panel (b) are fairly similar under both strategies.


\subsection{Child labor}
\label{sec:robustness3}

Child labor is a potential concern for my estimates.  While contemporary child labor rates in Indonesia are generally low, this was not true in the 1980s. The share of children aged 10-14 working declined from 11\% in 1980 to approximately 3\% in 2010.\footnote{Information about work is available only for people aged 10 or more.}	Moreover, the strong positive correlation between FLFP and female child labor (FCL) rates raises the possibility that birthplace effects could be driven by early labor market entry (See appendix figure \ref{fig:figure_B10}) 

However, my estimates are robust to controlling for FCL rates. Appendix figure \ref{fig:figure_B11} shows birthplace persistence estimates when controlling for birthplace FCL rates. The baseline estimates in orange (hollow circles) control for regency-year-age fixed effects, a quadratic polynomial on age, and education fixed effects. The estimates in red (filled circles) control for the birthplace FCL rate, while the purple estimates (plus sign markers) add interactions between migration age and the birthplace FCL rate. The estimates are largely unaffected by the inclusion of the child labor rates.


\subsection{Marriage-related migration}
\label{sec:marriage}

Marriage could drive the birthplace persistence in employment if there is an interaction between birthplace, migration age and marriage. Early marriage is associated with worse health and economic outcomes for women \citep{Corno2023}. If women from low-FLFP regencies are more likely to marry and migrate around 12-15 years old, this could explain why they are less likely to work later in life.

Appendix figure \ref{fig:figure_B12} uses detailed IFLS marriage history data to test whether marriage-related migration drives the employment patterns. First, in panel (a), I show the relationship between migration and women's marriage.\footnote{The IFLS collected marriage-history information for women only.} I classify migration episodes as marriage-related if the respondent married the year before, the year, or the year after she migrated. I then regress the marriage-related dummy on migration age fixed effects and interactions between migration age and birthplace FLFP. The plotted interaction estimates in panel (a) show a clear decline in the coefficients, suggesting that women from high-FLFP regencies become less likely to migrate due to marriage the longer they stay in their origin. This could explain the employment patterns I document. 

Nevertheless, panel (b) shows no evidence that selection on marriage-related migration accounts for the employment effects. Panel (b) displays the baseline IFLS estimates (red/hollow circles) along with results that control for interactions between migration age and the marriage-migration dummy and interactions between migration age, the marriage-migration dummy, and birthplace FLFP (orange/filled circles). If selection on marriage migration drove the birthplace effects, the trend should flatten once I account for the marriage motive. Nevertheless, the patterns remain virtually unchanged.




